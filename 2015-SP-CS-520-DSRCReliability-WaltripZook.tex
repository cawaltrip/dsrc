\documentclass[twoside,conference]{IEEEtran}
\usepackage[usenames,dvipsnames]{color}
\usepackage{hyperref}
\usepackage[nameinlink]{cleveref}

% PDF Metadata, Link Coloring
\hypersetup{
	pdftitle={Emergent Reliability Issues in Dedicated Short-Range Communication (DSRC)},
	pdfauthor={Chris Waltrip; Jared Zook},
	pdfsubject={University of Idaho; CS-520: Graduate Paper},
	pdfcreator={Chris Waltrip},
	pdfproducer={Chris Waltrip},
	linktoc=all, 			% Link the section number, text and page number in Contents
	colorlinks=false,        % Removes color frame and colors text instead
	linkcolor=ForestGreen,  % Default is red, may want to use black
	citecolor=Bittersweet,  % Default is green
	filecolor=Cyan,         % Default is cyan
	urlcolor=Magenta,       % Default is magenta
}

% Title Information
\title{Emergent Reliability Issues in Dedicated Short-Range Communication (DSRC)}
\author{
	\IEEEauthorblockN{Chris Waltrip}
	\IEEEauthorblockA{Department of Computer Science\\University of Idaho\\Moscow, Idaho 83843--1010\\Email: \href{mailto:walt2178@vandals.uidaho.edu}{\nolinkurl{walt2178@vandals.uidaho.edu}}}
	\and
	\IEEEauthorblockN{Jared Zook}
	\IEEEauthorblockA{Department of Computer Science\\University of Idaho\\Moscow, Idaho 83843--1010\\Email: \href{mailto:jzook@vandals.uidaho.edu}{\nolinkurl{jzook@vandals.uidaho.edu}}}
}

\begin{document}
\maketitle

\begin{abstract}
	This paper aims to identify reliability issues facing emergent Intelligent Transportation System (ITS) technology utilizing Dedicated Short-Range Communication (DSRC) radios for connected vehicles. Vehicles and infrastructure using DSRC technology form Vehicular Ad Hoc Networks (VANETs) which are prone to issues unique to the highly-dynamic, low-latency, limited bandwidth environments wherein they operate. This paper discusses noted reliability concerns with VANETs including congestion due to high traffic density, the hidden terminal problem, and weak points in the protocols that interact to allow communication between connected vehicles. It expands on the issues presented, discusses some proposed solutions to them, and suggests directions that future researchers may investigate\end{abstract}

\section{Introduction and Background}\label{sec:introduction}
	New wireless vehicular communication technology utilizing Dedicated Short-Range Communication (DSRC) allows for a host of new applications to make use of vehicle-to-vehicle (V2V) and vehicle-to-infrastructure (V2I) communication \cite{Kenney2011}. Benefits of DSRC communication include allowing for a reduction in traffic accidents, an increase in transportation efficiency, and the creation of a new a market for vehicular applications. To provide for these benefits, it is necessary to ensure that the technology is reliable at the wireless and application level. From a technical standpoint, the most important area of focus on DSRC reliability is on the behavior of DSRC wireless communications \cite{Bai2006}. Under this scheme, nodes communicate with each other directly, instead of through centralized stations, and form what is referred to as a Vehicular Ad Hoc Network (VANETs) \cite{Uhlemann2015}. DSRC VANETs are created on-the-fly when transmitting nodes (i.e. vehicles and/or infrastructure) are in close proximity to each other. The unique and dynamic nature of VANETs gives rise to many reliability questions, such as traffic congestion \cite{Jabbarpour2014}. This paper will focus in on some notable reliability issues facing vehicular communication using DSRC and will suggest what means may be implemented to overcome them.
	
	\subsection{WAVE/DSRC Background}\label{sec:wavedsrcbackground}
		% (Chris): This paragraph is clunky and needs to be rewritten.
		Wireless Access in Vehicular Environments (WAVE) and Dedicated Short-Range Communication (DSRC) are specified in IEEE Std. 1609 and IEEE Std 802.11p, respectively.  The IEEE Std. 1609.0-2010 describes the WAVE protocol stack using two planes, the management plane and the data plane \cite[p. 16]{IEEE2014-Std1609.0-2013}.  The data plane defines a shared set of physical (PHY), WAVE medium access control (MAC) and logical layer control (LLC) protocols as well a set of dual protocol stacks; the LLC and dual protocol stacks are defined in IEEE Std 1609.3 \cite[pp. 16--17]{IEEE2014-Std1609.0-2013}.  The first of the dual protocol stacks uses IPv6 and above layers (such as TCP and UDP) and the second of the stacks uses WAVE Short Message Protocol (WSMP) \cite[p. 9]{IEEE2010-Std1609.3-2010}.  Both of these protocol stacks accept data from higher layer protocols for transport using their respective protocols and also take data received and transmit it to higher network layers \cite[p. 9]{IEEE2010-Std1609.3-2010}.  Security features for WAVE are defined in IEEE Std. 1609.2 \cite{IEEE2013-Std1609.2-2013}.
	
		DSRC was originally standardized by ASTM International in ASTM E2213-03, which is based on the Orthogonal Frequency Division Multiplexing (OFDM) PHY and Carrier Sense Multiple Access/Collision Avoidance (CSMA/CA) MAC of IEEE 802.11a \cite[p. 10]{Lansford2013}.  The automotive industry eventually formalized these standards by amending IEEE 802.11, resulting in 802.11p \cite[p. 10]{Lansford2013}.  While there weren't many changes at the PHY layer, the MAC layer was changed to focus on making sure that the time required in order for two devices to communicate was as low as possible \cite[p. 10]{Lansford2013}\cite{Serageldin2013}.  That is to say, DSRC is the ``ad hoc part of WAVE \cite[p. 9]{Uhlemann2015}.''
	
	\subsection{Outline}\label{sec:outline}
		In \cref{sec:congestioncollision}, issues stemming from collision and congestion are discussed. \Cref{sec:hiddenterminal} then focuses on the hidden terminal problem and various MAC protocols that have been proposed to solve it.  Then in \cref{sec:coexistence}, issues that would be caused by allowing unlicensed devices into the DSRC bandwidth are examined.  \Cref{sec:beacons} focuses on the impact of beacon messages interfering with Basic Safety Messages.  Finally, \cref{sec:conclusions} contains conclusions and an analysis of the way the current research topics interact with each other.

\section{Congestion and Collision Control}\label{sec:congestioncollision}
	Controlling network congestion is a significant issue facing DSRC. At high traffic densities, many researchers have noted congestion problems including packet collisions and longer delays between successful receipt of packets. These collisions and delays compromise the effects of safety applications \cite{Subramanian2012}.
	
	One major reliability concern for safety message propagation in connected vehicles is packet loss due to collisions. When packets collide, they are lost and their data does not get transmitted, posing a hazard to drivers \cite{Hassan2011}. This section investigates problems stemming from collision and touches on the hidden terminal problem and the idea of bandwidth coexistence. 
	
	The Basic Safety Message (BSM) generation rate should be maintained at the maximum possible level so that reliability is guaranteed [or close to guaranteed] \cite{Wenfeng2014}.  Pourmohammadi investigates fairness using power control and range control and two algorithms that do this \cite{Pourmohammadi2015}.  Wenfeng mentions that Xu and Barth a rate control method is used \cite{Wenfeng2014,Xu2004-2}.  Kolte gives an overview of different congestion control protocols that have been proposed such as VeMAC, CCC-MAC and others as well as a proposed model for adaptive congestion control \cite{Kolte2014}. \\

	\subsection{Channel Width}\label{sec:channelwidth}
		\cite{Kenney2011} suggests that congestion control may be better addressed by utilizing 20 MHz channels instead of the 10 MHz channels DSRC is most often tested on. Then, the probability of a collision occurring diminishes because frames may be read at double the rate under certain schemes. However, this comes at the cost of having the channels becoming more susceptible to noise, which could also compromise reliability. Further, the 10 MHz width is better suited for issues arising in vehicular environments, e.g. delay, and is more commonly used.

	\subsection{Distributed Congestion Control}\label{sec:distributedcongestion}
		Uhlemann talks about distributed congestion control and that ETSI proposes this in ITS-G5\cite{Uhlemann2015}.  ETSI PHY and MAC layer document \cite{ETSI-ES202663} points to ETSI Technical Specification for Decentralized Congestion Control document (ETSI TS 102 687) for this\cite{ETSI-TS102687}.  Aygun et. al propose a transmit power control and rate control, specifically ECPR (Environment and Context Aware Combined Power and Rate) Distributed Congestion Control for Vehicular Communications\cite{Aygun2015}.
		
	\subsection{Visible Light Communication}\label{sec:visiblelightcommunication}
		The unreliability of DSRC communication is pronounced in high traffic density situations, such as the ones found in busy cities and on highways where many radios will be transmitting simultaneously in close proximity \cite{Cailean2014}. \cite{Cailean2014} suggests that the technology should be augmented with other technologies to increase reliability under heavy packet transmission. They offer up Visible Light Communication (VLC) as a possible solution. VLC operates by sending data between vehicles using light rays and sensors. This communication allows for a direct line of sight between vehicles that are in close range, e.g. in bumper-to-bumper traffic, and thus a reliable link between them. However, VLC is negatively effected at longer ranges due to phenomena such as refraction where the emitted light will bounce off into an unintended direction. When used in tandem with DSRC, the short-range benefits of VLC could offset the load on DSRC under heavy traffic by lessening the burden to send as many safety messages, dramatically reducing the congestion problem.

\section{Hidden Terminals}\label{sec:hiddenterminal}
		During safety message transmission from one vehicle, other vehicles may move out of range of the sending vehicle, missing out on the safety information, thereby posing a threat to safety. Hidden terminals are vehicles that have moved out of this range, but still share the range with terminals in between them. Since safety messages use a single-hop, DSRC messages are particularly susceptible to this issue \cite{Ma2009}.  The reason for this is that the MAC as defined in 802.11p does not use Request-to-Send (RTS) and Clear-to-Send (CTS) messages and instead uses one-hop broadcast messages to send emergency and safety messages \cite[p. 969]{Rahman2014}. 
		
		\subsection{VeMAC}\label{sec:vemac}
			Some researchers have developed special MAC protocols for VANET situations in an effort to solve the hidden terminal problem. One such protocol, VeMAC, uses multi-channel time-division multiplexing to assign disjoint time slot sequences to vehicles traveling in opposing directions. Its aim in doing so is to decrease or eliminate collisions, increasing control channel throughput and eliminating hidden terminals\cite{Omar2013}. This drawback of this protocol is that would require each node to transmit data for each of its time slots, even if it is not utilizing a high-priority application, increasing network congestion \cite{Kolte2014}. 
		
		Although different MAC protocols have been suggested, the effective protocol in use is the 802.11 DCF MAC protocol which was adopted in the IEEE 802.11p standard for DSRC communication \cite{Hassan2011}.
		
\section{Bandwidth Coexistence}\label{sec:coexistence}
		In September 2011, the Intelligent Transportation Services of America (ITSA) met with Congress to discuss bill that had been proposed that, if approved, would jeopardize the reliability of DSRC by opening up the 5.9 GHz bandwidth to unlicensed uses, such as from the 5.8 GHz 802.11a/n/ac (Wi-Fi) band \cite{ITSA2011,Lansford2013}.  In 2012, Congress requests that the FCC allow for unlicensed devices to be used in the DSRC range \cite{ITSA2014}.  They specifically request that the National Telecommunications \& Information Administration (NTIA) prepare a report about the methods for sharing bandwidth between DSRC and unlicensed WLAN devices and risks to Federal users if this were allowed \cite{HR2012}.  
		
		The NTIA report that was released in January 2013 found four risks that existed to DSRC.  Two of these risks are that current WLAN and DSRC devices aren't developed to detect the signals of the other. The third risk is that WLAN devices are meant to have co-located transmitters and receivers that DSRC doesn't require and that the hidden terminal problem exists in DSRC. The last risk is that Dynamic Frequency Selection (DFS), used in WLAN devices may not be able to detect DSRC signals and may cause serious interference with the DSRC communication \cite[pp. 53--54]{NTIA2013}.  In response to this report, the Regulatory Standing Committee of the 802.11 working group creates the DSRC Coexistence Tiger Team to also evaluate reliability issues stemming from sharing bandwidth with Unlicensed National Information Infrastructure (U-NII) devices, which are devices used by 802.11a in the 5 GHz band \cite[p. 14]{Lansford2013}. This team exploring the fundamental issue of how to share the bandwidth in a fair way considering that DSRC is to have priority over U-NII communications in the band \cite[p. 5]{Lansford2014}.
		
		 It appears that telecommunications and non-automotive companies such as Time Warner Cable \cite[p. 13--14]{TWC2013}, Google, Microsoft \cite[p. 11]{MicrosoftGoogle2013} and Comcast \cite[p. 30]{Comcast2013} support the coexistence of Wi-Fi and DSRC, while the automotive community is nervous that the coexistence will be detrimental to the reliability of DSRC \cite[p. 16]{Toyota2013}.
		 
		 \subsection{Priority Reversal}
		 \cite{Park2014} identifies a solution for solving the priority reversal issue between WAVE devices and U-NII-4 devices, which they refer to as WLAN devices.  The idea of priority reversal is that a WLAN device near traveling vehicles is transmitting information.  A WAVE device in one vehicle notices and delays sending its own BSMs until it has space in the band to do so, which in turn means that other vehicles won't receive messages.  The WAVE device should have priority, but in this scenario the WLAN device does.  There is a solution to this problem that can be implemented within the 802.11 standard.  This involves the adjustment of the arbitration inter-frame spacing number (AIFSN) of the WLAN device.  Without adjustment of the AIFSN, a WLAN device can cause the time that it takes for a WAVE device to transmit a BSM to increase by a factor of 40.  By adjusting this number, found the transmission time change due to the WLAN device becomes negligible \cite[p. 165]{Park2014}.
		 
		 % TODO (Chris): Add hidden terminal information to this section
		
\section{The Impact of Beacon Messages}\label{sec:beacons}
		Different locales may send beacon messages to apprise vehicles of local status information. Unfortunately, these messages are sent out over the control channel at various intervals. Since safety applications use the control channel, the possibility exists that beacon messages may collide with safety messages and the resultant packet loss could potentially create a safety hazard \cite{Doukha2015}. \cite{Doukha2015} suggests a number of solutions to this threat including implementing a dynamic beacon sending rate which could send fewer beacon messages under potentially dangerous scenarios and better synchronizing when beacons are sent out.

\section{Conclusions}\label{sec:conclusions}
	This paper discussed a wide array of reliability concerns stemming from the implementation of DSRC for use in connected vehicles and infrastructure. The standards and protocols put in place for this technology allow for efficient creation of VANETs which will spur many opportunities for development of safety and commercial applications for connected vehicles. However, as with any set of standards, trade-offs must be made in implementation. In the case of DSRC, these compromises open up major concerns regarding the reliability of the wireless network. The main issues in analyzing DSRC reliability come down to what will happen in hypothetical situations where inclement weather, road hazards, and high traffic densities will put reliable transmission of packets to the test. In addition to those concerns, corporate encroachment on the safety buffer sitting below the 5.9 GHz band raises issues about how different technologies can share bandwidth and, if the need arises, give way to higher priority communications. To address these issues, researchers and industry are actively developing new algorithms, protocols, technologies, and policies. 	Some of these developments hold up well in theory, but are highly unlikely or unfeasible in implementation. Several solutions proposed in this paper serve as an example of this effect. For example, many researchers have focused on developing better MAC protocols to eliminate the hidden terminal and congestion problems and many publications speak to this. Since these protocols differ from those laid out in the standard, they likely may not be worth looking into further. This is because of the different stakeholders involved in evolving DSRC technology need to adhere to the standard so that they are on the same page. Consistency in communication methods between disparate groups developing under this scheme is vital to the interoperation of systems, especially in systems like ITS where transportation is involved and lives will be at stake. This is what standards are for. Along similar lines, using wider channels could address some issues, but the idea has been set aside.  While new protocols and channel use methods may theoretically address some issues, the practical reality of implementing DSRC will, by and large, require different solutions.
	More feasible solutions for DSRC reliability concerns that this paper notes include developing better algorithms, developing new technologies to augment DSRC, and understanding further whether sharing bandwidth with other services may compromise DSRC reliability. Developing better algorithms should always be a focus in implementation, especially if they are well-tested and work in practice. It appears that there is an opportunity here for further research. Augmenting DSRC with additional technologies such as VLC is promising for lightening the load in congested situations and merits further research. The concern here, though, is that new technologies take time and money to develop and test, which could pose a major obstacle to sway stakeholders such as automakers who will have to invest even more than they already have in developing DSRC systems themselves. Finally, bandwidth coexistence between DSRC and other services sharing portions of the same band offers an opportunity for further research. If services sharing bandwidth with DSRC safety messages can reliably yield control in emergency situations, this would ease many concerns with the idea of sharing the spectrum.
	

\bibliographystyle{IEEEtranS}
\bibliography{IEEEabrv,./IEEE-BSTcontrol,./CS520-Bibliography}

\end{document}























