\documentclass[twoside,conference]{IEEEtran}
\usepackage[usenames,dvipsnames]{color}
\usepackage{hyperref}
\usepackage[nameinlink]{cleveref}


% TODO (Chris): Change colorlinks back to false before submitting
% PDF Metadata, Link Coloring
\hypersetup{
	pdftitle={Emergent Issues in Dedicated Short-Range Communication (DSRC) Reliability},
	pdfauthor={Chris Waltrip; Jared Zook},
	pdfsubject={University of Idaho; CS-520: Graduate Paper},
	pdfcreator={Chris Waltrip},
	pdfproducer={Chris Waltrip},
	linktoc=all, 			% Link the section number, text and page number in Contents
	colorlinks=true,        % Removes color frame and colors text instead
	linkcolor=ForestGreen,  % Default is red, may want to use black
	citecolor=Bittersweet,  % Default is green
	filecolor=Cyan,         % Default is cyan
	urlcolor=Magenta,       % Default is magenta
}

% Title Information
\title{Emergent Reliability Issues in Dedicated Short-Range Communication (DSRC)}
\author{
	\IEEEauthorblockN{Chris Waltrip}
	\IEEEauthorblockA{Department of Computer Science\\University of Idaho\\Moscow, Idaho 83843--1010\\Email: \href{mailto:walt2178@vandals.uidaho.edu}{\nolinkurl{walt2178@vandals.uidaho.edu}}}
	\and
	\IEEEauthorblockN{Jared Zook}
	\IEEEauthorblockA{Department of Computer Science\\University of Idaho\\Moscow, Idaho 83843--1010\\Email: \href{mailto:jzook@vandals.uidaho.edu}{\nolinkurl{jzook@vandals.uidaho.edu}}}
}

\begin{document}
\maketitle

\begin{abstract}
	This paper aims to seek reliability issues facing emergent Intelligent Transportation System (ITS) technology utilizing Dedicated Short-Range Communication radios. These radios form Vehicular Ad Hoc Networks (VANETs) which are prone to issues unique to the highly-dynamic, low-latency, limited bandwidth environments wherein they operate. This paper discusses problems with congestion due to high traffic density, the hidden terminal problem, and the impact of high relative speeds of vehicles transmitting messages. Then, it suggests further directions research should focus on.
\end{abstract}

\section{Introduction and Background}
\label{sec:introduction}
	New wireless vehicular communication technology utilizing Dedicated Short-Range Communication (DSRC) allows for a host of new applications to make use of vehicle-to-vehicle (V2V) and vehicle-to-infrastructure (V2I) communication \cite{Kenney2011}. Benefits of DSRC communication include allowing for a reduction in traffic accidents, an increase in transportation efficiency, and the creation of a new a market for vehicular applications. To provide for these benefits, it is necessary to ensure that the technology is reliable at the wireless and application level. From a technical standpoint, the most important area of focus on DSRC reliability is on the behavior of DSRC wireless communications \cite{Bai2006}. Under this scheme, nodes communicate with each other directly, forming what is referred to as a Vehicular Ad Hoc Network (VANETs) \cite{Uhlemann2015}. DSRC VANETs are created on-the-fly when transmitting nodes (i.e. vehicles or infrastructure) are in close proximity to each other. The unique nature of VANETs give rise to many reliability questions, such as traffic congestion \cite{Jabbarpour2014}. This paper will focus in on some notable reliability issues facing vehicular communication using DSRC and will suggest what means may be implemented to overcome them. \\

	Some of the topics to be explored include:
	\begin{enumerate}
		\item Cooperative Intelligent Transport Systems - Two wireless technologies: DSRC and 3G
		\item Distributed Congestion Control - ITS-G5 based on 802.11p; transmit \{rate,power,data rate\} control \cite{Uhlemann2015}
		\item Collisions \cite{Wenfeng2014}
	\end{enumerate}

	% TODO (Chris): Define and expand on these terms.
	A number of these topics are defined well in \cite{Serageldin2013}.
	\begin{enumerate}
		\item VANET
		\item C-ITS
		\item Congestion vs Collisions (at least in this paper)
	\end{enumerate}
	
	\subsection{WAVE/DSRC Background}
		% TODO (Chris): This paragraph is clunky and needs to be rewritten.
		Wireless Access in Vehicular Environments (WAVE) and Dedicated Short-Range Communication (DSRC) are specified in IEEE Std. 1609 and IEEE Std 802.11p, respectively.  The IEEE Std. 1609.0-2010 describes the WAVE protocol stack using two planes, the management plane and the data plane \cite[p. 16]{IEEE2014-Std1609.0-2013}.  The data plane defines a shared set of physical (PHY), WAVE medium access control (MAC) and logical layer control (LLC) protocols as well a set of dual protocol stacks; the LLC and dual protocol stacks are defined in IEEE Std 1609.3 \cite[pp. 16--17]{IEEE2014-Std1609.0-2013}.  The first of the dual protocol stacks uses IPv6 and above layers (such as TCP and UDP) and the second of the stacks uses WAVE Short Message Protocol (WSMP) \cite[p. 9]{IEEE2010-Std1609.3-2010}.  Both of these protocol stacks accept data from higher layer protocols for transport using their respective protocols and also take data received and transmit it to higher network layers \cite[p. 9]{IEEE2010-Std1609.3-2010}.  Security features for WAVE are defined in IEEE Std. 1609.2 \cite{IEEE2013-Std1609.2-2013}.
	
		% TODO (Chris): Expand on DSRC more; we don't have access to ATSM ES2213, which seems to be the standard that defines DSRC
		DSRC was originally standardized by ASTM International in ASTM E2213-03, which is based on the Orthogonal Frequency Division Multiplexing (OFDM) PHY and Carrier Sense Multiple Access/Collision Avoidance (CSMA/CA) MAC of IEEE 802.11a \cite[p. 10]{Lansford2013}.  The automotive industry eventually formalized these standards by amending IEEE 802.11, resulting in 802.11p \cite[p. 10]{Lansford2013}.  While there weren't many changes at the PHY layer, the MAC layer was changed to focus on making sure that the time required in order for two devices to communicate was as low as possible \cite[p. 10]{Lansford2013}\cite{Serageldin2013}.  That is to say, DSRC is the ``ad hoc part of WAVE \cite[p. 9]{Uhlemann2015}.''
	
	\subsection{Outline}
		In \cref{sec:congestion}, issues stemming from congestion and congestion control are discussed. \Cref{sec:collision} discusses reliability issues caused by collision and collision control.  The paper ends with conclusions in \cref{sec:conclusions}.

\section{Congestion Control}\label{sec:congestion}

	Controlling congestion is a significant issue facing DSRC. At high traffic densities, many researchers have noted congestion problems including packet collisions and longer delays between successful receipt of packets. These collisions and delays compromise the effects of safety applications \cite{Subramanian2012}.
	
	The Basic Safety Message (BSM) generation rate should be maintained at the maximum possible level so that reliability is guaranteed [or close to guaranteed] \cite{Wenfeng2014}.  Pourmohammadi investigates fairness using power control and range control and two algorithms that do this \cite{Pourmohammadi2015}.  Wenfeng mentions that Xu and Barth a rate control method is used \cite{Wenfeng2014,Xu2004-2}.  Kolte gives an overview of different congestion control protocols that have been proposed such as VeMAC, CCC-MAC and others as well as a proposed model for adaptive congestion control \cite{Kolte2014}. \\

	\subsection{Channel Width}\label{sec:channelwidth}
		\cite{Kenney2011} suggests that congestion control may be better addressed by utilizing 20 MHz channels instead of the 10 MHz channels DSRC is most often tested on. Then, the probability of a collision occurring diminishes because frames may be read at double the rate under certain schemes. However, the comes at the cost of the channels being more susceptible to noise. Further, the 10 MHz width is better suited for issues arising in vehicular environments, e.g. delay, and is more commonly used.
	
	\subsection{Hop Count}\label{sec:hopcount}
		Most high-priority safety applications are based off one-hop broadcasts \cite{Omar2013}. 

	\subsection{Distributed Congestion Control}\label{sec:distributedcongestion}
		Uhlemann talks about distributed congestion control and that ETSI proposes this in ITS-G5\cite{Uhlemann2015}.  ETSI PHY and MAC layer document \cite{ETSI-ES202663} points to ETSI Technical Specification for Decentralized Congestion Control document (ETSI TS 102 687) for this\cite{ETSI-TS102687}.  Aygun et. al propose a transmit power control and rate control, specifically ECPR (Environment and Context Aware Combined Power and Rate) Distributed Congestion Control for Vehicular Communications\cite{Aygun2015}.

\section{Collision Control}\label{sec:collision}
	One major reliability concern for safety message propagation in connected vehicles is packet loss due to collisions. When packets collide, they are lost and their data does not get transmitted, posing a hazard to drivers \cite{Hassan2011}. This section investigates problems stemming from collision and touches on the hidden terminal problem and the idea of bandwidth coexistence. 

	\subsection{Hidden Terminals}\label{sec:hiddenterminal}
		During safety message transmission from one vehicle, other vehicles may move out of range of the sending vehicle, missing out on the safety information, thereby posing a threat to safety. Hidden terminals are vehicles that have moved out of this range, but still share the range with terminals in between them. Since safety messages use a single-hop, DSRC messages are particularly susceptible to this issue \cite{Ma2009}.  The reason for this is that the MAC as defined in 802.11p does not use Request-to-Send (RTS) and Clear-to-Send (CTS) messages and instead uses broadcast messages to send emergency and safety messages \cite[p. 969]{Rahman2014}. 
		
		Some researchers have developed special MAC protocols for VANET situations in an effort to eliminate the hidden terminal problem. One such protocol, VeMAC, uses multi-channel time-division multiplexing to assign disjoint time slot sequences to vehicles traveling in opposing directions. Its aim in doing so is to decrease or eliminate collisions, increasing control channel throughput \cite{Omar2013}. This protocol would require each node to transmit data for each of its time slots, even if it is not utilizing a high-priority application \cite{Kolte2014}. 
		
		% TODO (Chris): Expand on ACC
		Another protocol is Adaptive Congestion Control as proposed by \cite[p. 3]{Kolte2014}
		
		Although different MAC protocols have been suggested, the IEEE 802.11p standard for DSRC communication has adopted the 802.11 DCF MAC protocol \cite{Hassan2011}. 
		
	\subsection{Bandwidth Coexistence}\label{sec:coexistence}
		In September 2011, the Intelligent Transportation Services of America (ITSA) met with Congress to discuss bill that had been proposed that, if approved, would jeopardize the reliability of DSRC by opening up the 5.9 GHz bandwidth to unlicensed uses, such as from the 5.8 GHz WiFi band \cite{ITSA2011,Lansford2013}.
	
		Some of the sources to use here are \cite{Lansford2013,Chang2013,Toyota2013,NTIA2013}
		
	\subsection{The Impact of Beacon Messages}\label{sec:beacons}
		Different locales may send beacon messages to apprise vehicles of local status information. Unfortunately, these messages are sent out over the control channel at various intervals. Since safety applications use the control channel, the possibility exists that beacon messages may collide with safety messages and the resultant packet loss could potentially create a safety hazard \cite{Doukha2015}. \cite{Doukha2015} suggests a number of solutions to this threat including implementing a dynamic beacon sending rate which could send fewer beacon messages under potentially dangerous scenarios and better synchronizing when beacons are sent out.

\section{Conclusions}\label{sec:conclusions}
	Give a summary of the different things that were discussed in this paper.

% TODO (Chris): Remove this clearpage after the paper is done.
\cleardoublepage
\bibliographystyle{IEEEtranS}
\bibliography{IEEEabrv,./IEEE-BSTcontrol,./CS520-Bibliography}

\end{document}